%
% File acl2015.tex
%
% Contact: car@ir.hit.edu.cn, gdzhou@suda.edu.cn
%%
%% Based on the style files for ACL-2014, which were, in turn,
%% Based on the style files for ACL-2013, which were, in turn,
%% Based on the style files for ACL-2012, which were, in turn,
%% based on the style files for ACL-2011, which were, in turn, 
%% based on the style files for ACL-2010, which were, in turn, 
%% based on the style files for ACL-IJCNLP-2009, which were, in turn,
%% based on the style files for EACL-2009 and IJCNLP-2008...

%% Based on the style files for EACL 2006 by 
%%e.agirre@ehu.es or Sergi.Balari@uab.es
%% and that of ACL 08 by Joakim Nivre and Noah Smith

\documentclass[9pt]{article}
\usepackage{acl2015}
\usepackage{times}
\usepackage{url}
\usepackage{latexsym}

%\setlength\titlebox{5cm}

% You can expand the titlebox if you need extra space
% to show all the authors. Please do not make the titlebox
% smaller than 5cm (the original size); we will check this
% in the camera-ready version and ask you to change it back.


\title{NLP \& Machine Learning Applied: Video Game Reviews}

\author{Matthew D. Mulholland \\
  Montclair State University \\
  Montclair, NJ \\
  {\tt mulhollandm2@montclair.edu}}

\date{}

\begin{document}
\maketitle
\begin{abstract}
Despite the ambiguity of the concept, much research has been done in the area of detecting ``fake'' reviews. It is, however, often difficult to build corpora containing reviews that are definitively ``fake'' or ``true''. It is simpler to reframe the issue in terms of the amount of experience a reviewer has with a product: given a review, can we tell anything about the level of experience the reviewer has with the reviewed product?

In this exploratory paper, I will detail a research project whose aim was to relate video game reviews to proxies for reviewer experience, such as number of hours played, number of times marked as helpful, and other related review/user attributes, using natural language processing and machine learning techniques. The ultimate end is to produce a capability for ranking or filtering reviews, which could be used in addition to or in place of other fake review or spam filtering algorithms.

A less grand aim of the project was to scrape review data from the Steam video game website and make it publicly available. This data will be described at length.
\end{abstract}

\section{Credits}

I would like to thank both Janette Martinez and Emily Olshefski for their help in the initial iteration of this work as part of a class project. They helped lay out the problem, make decisions regarding the source of the data and the games for which data was collected, and also write some of the preprocessing code. Some sections of this paper build on the final paper for that class project, which they took a lead in writing.

\section{Introduction}

\subsection{Reframing a familiar problem}

In the realm of deception detection, ground truth -- information that can be verified or denied -- is not easy to come by. For example, one application of deception detection is in the detection of fake reviews: reviews that are known to be fake (by some external means) are compared to reviews that are believed to have been written in good faith. It may be easy in some cases to identify certain reviews as fake since the authors themselves might admit as much. However, the categorization of other reviews from either category is a difficult matter. Thus, corpora of fake/real reviews are often constrained in that, for any given review, it could be impossible to determine the category.

In this paper, I propose a fundamental reframing of the issue: the problem should not be about whether or not a given review is fake, but rather about measuring the amount of experience a reviewer has with the product being reviewed. Reviews for products of which the reviewer has little to no experience -- whether they have been written in bad faith or simply from a relatively uninformed perspective -- could be distinguished from reviews for which the reviewer does have experience with the product being reviewed. Further, different levels of experience -- say, moderate or high -- could be distinguished from one another.

A system that could accurately predict the amount of experience a reviewer has given only the review text or some combination of the review text and other attributes of the review/reviewer, such as the number of times a review has been marked as helpful, could be useful in a production environment as a component in a review-filtering and/or -sorting algorithm.

\subsection{How is experience measured?}

To model reviewer experience (which is, after all, just as nebulous as the distinction between fake and true reviews), there must be methods of approximating experience. One simple way to approximate this value is by recording the amount of time a user has actually spent using a product before reviewing it. For most products, however, the feasibility of recording the actual time spent is low due to the cost and logistics of such an undertaking. And that is assuming time even can be recorded! In other cases, product use is more of a binary value: Did the consumer eat the pie or not? Did the user watch the movie or not? And, in still other cases, usage might need to be measured in terms of the number of times the product was used, such as lotion, hair spray, etc.

Aside from trying to measure a reviewer's actual usage of a product, there may exist other ways of approximating experience. One rather indirect way would be to compare the number of times each review is marked as helpful (or not helpful). Presumably, reviews that are more helpful are reviews that are from users who actually used the product and vice versa.

\subsection{Using video game reviews from Steam}

In the realm of video games, the methods of approximating user experience described above are actually feasible. The Steam online video game platform (\url{http://store.steampowered.com/}) is used by video game enthusiasts for a variety of purposes. Steam functions as a platform

\begin{itemize}
\item in which users can play video games in an online, social environment
\item from which users can purchase, research, and review video games
\item for social iteraction (through video game reviews, marking reviews as helpful, funny, etc., displaying playing stats and achievements, etc.)
\end{itemize}

Steam keeps a record of the number of hours each user has played each game and, thus, when a user submits a review of a video game, this information is presented alongside the review. It is true that the meaning of this measure is not completely straightforward: a user could have played a particular game for many hours outside the Steam platform and these hours would not be included. However, it is a reasonable assumption to make that the majority of the values recorded by the platform will be accurate and/or that the amount of time that players have played a particular game outside the confines of the online platform will be similar across players submitting reviews for the same game.

The window into a user's experience with a product that is afforded by the amount of time a user has used a product is somewhat unique to the case of video games: for other products, it might be impossible or pointless to attempt to record the amount of time the user used the product. For example, there is no way to keep a record of the amount of time a user has spent using a vacuum short of conducting a study and painstakingly recording such information.

However, if experience could successfully be modelled in the case of video games, perhaps the models could be generalized to cover whole categories of products and, thus, the situation here would apply to much more than video games. Furthermore, there are other indications of a user's experience with a product, as mentioned above, such as the number of times a review has been marked as helpful, and fortunately Steam also keeps a record of such information.

\subsection{Description of research}

In this paper, I explore the amount of time a reviewer spent playing the game he/she is reviewing, the number of times it was marked helpful by review readers, and other attributes of the review in relation to review text itself (and also to a set of attributes about the review/reviewer, such as the reviewer's number of friends, the number of times the review was marked as funny, etc.). The greater the amount of experience a reviewer has with the game being reviewed, the greater the likelihood is that the review produced is trustworthy or in some sense valuable. Naturally, if a reviewer has spent a lot of time playing a game or if the review is voted to be relatively helpful, etc., then the review has a higher chance of representing a good understanding of the game on the part of the reviewer.

Fortunately, the online gaming platform Steam collects such data about its users and makes it publicly accessible. A web-scraping method was developed and used to build a corpus of review texts along with a lot of additional data, as mentioned briefly above. Reviews from 11 of the most popular games were scraped from the Steam website. After filtering out non-English reviews, the data was partitioned into training/test sets. A number of commonly-used NLP feature types were extracted from the reviews and machine learning experiments were conducted using a range of learning algorithms.

The main motivation of the machine learning experiments was to determine whether or not the various proxies for experience under consideration demonstrated potential in terms of whether or not they could successfully be modelled. The results are mixed, but there is some indication that this is a worthy research effort and that it could lead down the road to some of the grand aims mentioned above, i.e., influencing a review filtering/sorting algorithm.

% For bold text:
%\textbf{The proceedings are designed for printing on A4 paper.}

% For URLS:
%We will make more detailed instructions available at
%\url{http://acl2015.org/publication.html}. Please check this website 
%regularly.

\section{Related Works}

While this paper does not fall under the purview of typical deception detection work, influences from other deception detection work form the basis for some of the underlying ideas of this research. ~\cite{Jindal:08} used product review data from {\tt Amazon.com} to find ``opinion spam''. Discovering opinion spam led to research into deceptive reviews. Since there was no gold standard data to work from at that time, they initiated detection of such spam by first detecting duplicate reviews and then by using supervised learning with manually-labeled training examples. They gathered reviews from different categories such as music, books, and DVDs, but video games were not specifically included. Amazon reviews were also used by~\cite{Forn:14} to test identification of fake reviews using crowdsourcing.

~\cite{Ott:11} developed gold standard deception review data consisting of 400 true and 400 false reviews. The 400 fake reviews were written by Amazon Mechanical Turkers while the real ones were mined from {\tt Trip Advisor.com } (and pruned to match the criteria given to the turkers). Naive Bayes, Support Vector Machines(SVM LIGHT), and the Standford Parser were used and an accuracy of 89.6\% was achieved using bigram features. ~\cite{Feng:12a}, using Ott's gold-standard corpus, extracted PCFG parse trees for deep syntax encoding along with some shallower syntactic/semantic features (such as LIWC features and POS tags to garner comparisons of effectiveness) in addition to the original bigram features. With support vector machine, they were able to improve on Ott's accuracy score, increasing it to 91.2\%.

One advantage of using data from Steam (as opposed to {\tt Amazon.com} or {\tt TripAdvisor.com}, both of which have stored tranasctional data) is that Steam's data goes a step beyond the ground truth of the aforementioned websites. Not only is transactional data easily accessible, but the fact is that hours played values provide a window to the ground truth of product experience. There is no practical litmus test for the prior works to test trustworthiness in the same way. Rather, prior works have had to resort to circuitous ways of testing the trustworthiness of reviews.

\section{Data and Code}

\subsection{Data}

Data from 11 popular video games was scraped from the Steam website via a web scraping program written in Python. See Table \ref{tab:datasets} for a full listing of the games and the number of reviews that were collected for each game (after filtering of non-English reviews, reviews that had zero content, etc.). The program was designed to scrape the following (non-exhaustive) types of information from reviews:

\begin{itemize}
\item review text
\item amount of time reviewer spent playing game (in hours and to one decimal place)
\item number of times a review was marked helpful
\item number of times a review was marked unhelpful
\item number of times a review was marked funny
\item number of comments other users have left in the review's discussion space
\item number of friends the reviewer has on the Steam platform
\item number of Steam groups the reviewer is a member of
\item number of reviews, guides, and screenshots the reviewer has posted (for this and other games)
\item number of Steam ``badges'' the reviewer has received
\item number of in-game achievements the reviewer has attained
\end{itemize}

All data-sets have been made freely available with a permissive license (MIT) via {\tt GitHub.com}. The data can be found at the following URL: \url{https://github.com/mulhod/steam_reviews}. It is stored in relatively common and portable {\tt JSONLINES/NDJ} format. The data was collected over a period of approximately two weeks in the summer of 2015.

\begin{table}[ht]
\centering
\renewcommand{\arraystretch}{1.2}
\begin{tabular}{|l|c|}
\hline \bf Game & \bf \# reviews \\ \hline
Arma 3 & 7,151 \\
Counter Strike & 6,040 \\
\parbox[t]{4cm}{Counter Strike: Global\\Offensive} & 7,073 \\
Dota 2 & 9,720 \\
Football Manager 2015 & 1,522 \\
Garry's Mod & 7,151 \\
Grand Theft Auto V & 13,349 \\
Sid Meier's Civilization 5 & 7,467 \\
Team Fortress 2 & 5,676 \\
The Elder Scrolls V & 7,165 \\
Warframe & 7,123 \\ \hline
\end{tabular}
\caption{Review data-sets. Shows the name of each game and how many reviews were collected.}
\label{tab:datasets}
\end{table}

\subsection{Code}

Like the data, all code written for and used in this project has been made freely available via {\tt GitHub.com}. The code repository can be found at the following URL: \url{https://github.com/mulhod/reviewer_experience_prediction}. Also provided are in-depth descriptions of the data, full ReST-style documentation for the code, clear instructions on how to get set up and running, and numerous Jupyter (formerly known as IPython) notebooks describing various algorithms and aspects of the code and providing visualizations of the data, etc. All code is written in the Python programming language and the total number of lines of code exceeds 6,000. Cython is used to convert large parts of the code base to C extensions to improve performance. The {\tt spaCy} Python library is used for most of the natural language processing (NLP) analysis.

\section{Methods}

After collecting reviews from the Steam website, the reviews were analyzed and a set of NLP features were extracted. All NLP features and review/reviewer attributes were stored in a MongoDB database collection. MongoDB is particularly useful in cases where the data is unstructured: it allows for easily inserting new fields for existing document entries. In the case of this project, the amount of data and the time it takes to extract NLP features for each review text necessitated some creative thinking in terms of how best to store the data and reduce the computational load/the time needed to conduct many different experiments with potentially the same data. It was determined that a database holding all reviews and all NLP features would be the best solution despite the fact that the database would need a lot of memory devoted to it -- the database ended up needing over 200 GiB of hard-drive space! The benefits of having a running database are that the data can be accessed at any time and that it can be queried in a way that would be impossible if it existed in flat files.

\subsection{NLP feature extraction}

Using the {\tt spaCy} Python library for NLP analysis and other tools, all review texts were analyzed and NLP features were extracted. The following types of NLP features were extracted:

\begin{itemize}
\item word $n$-grams for $n$ = 1 to 2
\item character $n$-grams for $n$ = 2 to 5
\item syntactic dependency relationships
\item review length ($log_2 x$ where $x$ refers to the number of characters in the review)
\item Brown corpus cluster IDs (for each word in the review text)
\end{itemize}

For the first feature type in the list above, otherwise known as ``the bag-of-words'' method, the review text is lower-cased and tokenized and then all one- and two-word sequences are identified. Likewise, for the character $n$-grams features, all two-, three-, four-, and five-character sequences are extracted from the raw, unprocessed review text. This latter feature type serves a couple different purposes:

\begin{itemize}
\item It extracts some textual features that would not be extracted via the $n$-gram features, such as emoticons, sequences of non-word (and word) characters, etc.
\item It implicitly implements a basic form of automatic spelling correction as even misspelled tokens will still have large sub-strings in common with the correctly-spelled versions.
\end{itemize}

The third type of extracted feature, syntactic dependencies, extends the ``bag-of-words'' approach. The $n$-gram features represent sequences of characters/tokens and, for some simple cases (in English, at the very least), this suffices to extract a full syntactic structure. However, when syntactic structures are not sequential/linear or there is intervening content, such as embedded clauses, modifiers, etc., or the relationship simply manifests itself over a long stretch of words, the $n$-gram features are not enough. The syntactic dependency features are meant to cover such phenomena.

Lastly, the review length (as a base-2 logarithm of the number of characters, whose slope will decrease with review length) and a set of Brown corpus cluster IDs are extracted from each review. Although it is expected that review length may correlate weakly with experience (the more a user has used a product, the more he/she may have to say about it), the relationship is not necessarily linear. The difference between a review that is 10 characters long versus a review that is 100 characters in length is perhaps more significant than the difference between a 5,000-character review and a 10,000-character review. The Brown corpus cluster IDs are intended to capture a crude type of vocabulary distribution feature. Each token's cluster ID refers to a particular cluster of texts from the Brown corpus. Representing each review as a set of cluster IDs should provide some abstract idea about the content of the review.

\section{General Instructions}

Manuscripts must be in two-column format.  Exceptions to the
two-column format include the title, authors' names and complete
addresses, which must be centered at the top of the first page, and
any full-width figures or tables (see the guidelines in
Subsection~\ref{ssec:first}). {\bf Type single-spaced.}  Start all
pages directly under the top margin. See the guidelines later
regarding formatting the first page.  The manuscript should be
printed single-sided and its length
should not exceed the maximum page limit described in Section~\ref{sec:length}.
Do not number the pages.


\subsection{Electronically-available resources}

We strongly prefer that you prepare your PDF files using \LaTeX\ with
the official ACL 2015 style file (acl2015.sty) and bibliography style
(acl.bst). These files are available at
\url{http://acl2015.org}. You will also find the document
you are currently reading (acl2015.pdf) and its \LaTeX\ source code
(acl2015.tex) on this website.

You can alternatively use Microsoft Word to produce your PDF file. In
this case, we strongly recommend the use of the Word template file
(acl2015.dot) on the ACL 2015 website (\url{http://acl2015.org}). 
If you have an option, we recommend that you use the \LaTeX2e version. 
If you will be using the Microsoft Word template, we suggest that you 
anonymize your source file so that the pdf produced does not retain your
identity.  This can be done by removing any personal information
from your source document properties.



\subsection{Format of Electronic Manuscript}
\label{sect:pdf}

For the production of the electronic manuscript you must use Adobe's
Portable Document Format (PDF). PDF files are usually produced from
\LaTeX\ using the \textit{pdflatex} command. If your version of
\LaTeX\ produces Postscript files, you can convert these into PDF
using \textit{ps2pdf} or \textit{dvipdf}. On Windows, you can also use
Adobe Distiller to generate PDF.

Please make sure that your PDF file includes all the necessary fonts
(especially tree diagrams, symbols, and fonts with Asian
characters). When you print or create the PDF file, there is usually
an option in your printer setup to include none, all or just
non-standard fonts.  Please make sure that you select the option of
including ALL the fonts. \textbf{Before sending it, test your PDF by
  printing it from a computer different from the one where it was
  created.} Moreover, some word processors may generate very large PDF
files, where each page is rendered as an image. Such images may
reproduce poorly. In this case, try alternative ways to obtain the
PDF. One way on some systems is to install a driver for a postscript
printer, send your document to the printer specifying ``Output to a
file'', then convert the file to PDF.

It is of utmost importance to specify the \textbf{A4 format} (21 cm
x 29.7 cm) when formatting the paper. When working with
{\tt dvips}, for instance, one should specify {\tt -t a4}.
Or using the command \verb|\special{papersize=210mm,297mm}| in the latex
preamble (directly below the \verb|\usepackage| commands). Then using 
{\tt dvipdf} and/or {\tt pdflatex} which would make it easier for some.


Print-outs of the PDF file on A4 paper should be identical to the
hardcopy version. If you cannot meet the above requirements about the
production of your electronic submission, please contact the
publication chairs as soon as possible.


\subsection{Layout}
\label{ssec:layout}

Format manuscripts two columns to a page, in the manner these
instructions are formatted. The exact dimensions for a page on A4
paper are:

\begin{itemize}
\item Left and right margins: 2.5 cm
\item Top margin: 2.5 cm
\item Bottom margin: 2.5 cm
\item Column width: 7.7 cm
\item Column height: 24.7 cm
\item Gap between columns: 0.6 cm
\end{itemize}

\noindent Papers should not be submitted on any other paper size.
 If you cannot meet the above requirements about the production of 
 your electronic submission, please contact the publication chairs 
 above as soon as possible.


\subsection{Fonts}

For reasons of uniformity, Adobe's {\bf Times Roman} font should be
used. In \LaTeX2e{} this is accomplished by putting

\begin{quote}
\begin{verbatim}
\usepackage{times}
\usepackage{latexsym}
\end{verbatim}
\end{quote}
in the preamble. If Times Roman is unavailable, use {\bf Computer
  Modern Roman} (\LaTeX2e{}'s default).  Note that the latter is about
  10\% less dense than Adobe's Times Roman font.


\begin{table}[h]
\begin{center}
\begin{tabular}{|l|rl|}
\hline \bf Type of Text & \bf Font Size & \bf Style \\ \hline
paper title & 15 pt & bold \\
author names & 12 pt & bold \\
author affiliation & 12 pt & \\
the word ``Abstract'' & 12 pt & bold \\
section titles & 12 pt & bold \\
document text & 11 pt  &\\
captions & 11 pt & \\
abstract text & 10 pt & \\
bibliography & 10 pt & \\
footnotes & 9 pt & \\
\hline
\end{tabular}
\end{center}
\caption{\label{font-table} Font guide. }
\end{table}

\subsection{The First Page}
\label{ssec:first}

Center the title, author's name(s) and affiliation(s) across both
columns. Do not use footnotes for affiliations. Do not include the
paper ID number assigned during the submission process. Use the
two-column format only when you begin the abstract.

{\bf Title}: Place the title centered at the top of the first page, in
a 15-point bold font. (For a complete guide to font sizes and styles,
see Table~\ref{font-table}) Long titles should be typed on two lines
without a blank line intervening. Approximately, put the title at 2.5
cm from the top of the page, followed by a blank line, then the
author's names(s), and the affiliation on the following line. Do not
use only initials for given names (middle initials are allowed). Do
not format surnames in all capitals (e.g., use ``Schlangen'' not
``SCHLANGEN'').  Do not format title and section headings in all
capitals as well except for proper names (such as ``BLEU'') that are
conventionally in all capitals.  The affiliation should contain the
author's complete address, and if possible, an electronic mail
address. Start the body of the first page 7.5 cm from the top of the
page.

The title, author names and addresses should be completely identical
to those entered to the electronical paper submission website in order
to maintain the consistency of author information among all
publications of the conference. If they are different, the publication
chairs may resolve the difference without consulting with you; so it
is in your own interest to double-check that the information is
consistent.

{\bf Abstract}: Type the abstract at the beginning of the first
column. The width of the abstract text should be smaller than the
width of the columns for the text in the body of the paper by about
0.6 cm on each side. Center the word {\bf Abstract} in a 12 point bold
font above the body of the abstract. The abstract should be a concise
summary of the general thesis and conclusions of the paper. It should
be no longer than 200 words. The abstract text should be in 10 point font.

{\bf Text}: Begin typing the main body of the text immediately after
the abstract, observing the two-column format as shown in 
the present document. Do not include page numbers.

{\bf Indent} when starting a new paragraph. Use 11 points for text and 
subsection headings, 12 points for section headings and 15 points for
the title. 

\subsection{Sections}

{\bf Headings}: Type and label section and subsection headings in the
style shown on the present document.  Use numbered sections (Arabic
numerals) in order to facilitate cross references. Number subsections
with the section number and the subsection number separated by a dot,
in Arabic numerals. Do not number subsubsections.

{\bf Citations}: Citations within the text appear in parentheses
as~\cite{Gusfield:97} or, if the author's name appears in the text
itself, as Gusfield~\shortcite{Gusfield:97}.  Append lowercase letters
to the year in cases of ambiguity.  Treat double authors as
in~\cite{Aho:72}, but write as in~\cite{Chandra:81} when more than two
authors are involved. Collapse multiple citations as
in~\cite{Gusfield:97,Aho:72}. Also refrain from using full citations
as sentence constituents. We suggest that instead of
\begin{quote}
  ``\cite{Gusfield:97} showed that ...''
\end{quote}
you use
\begin{quote}
``Gusfield \shortcite{Gusfield:97}   showed that ...''
\end{quote}

If you are using the provided \LaTeX{} and Bib\TeX{} style files, you
can use the command \verb|\newcite| to get ``author (year)'' citations.

As reviewing will be double-blind, the submitted version of the papers
should not include the authors' names and affiliations. Furthermore,
self-references that reveal the author's identity, e.g.,
\begin{quote}
``We previously showed \cite{Gusfield:97} ...''  
\end{quote}
should be avoided. Instead, use citations such as 
\begin{quote}
``Gusfield \shortcite{Gusfield:97}
previously showed ... ''
\end{quote}

\textbf{Please do not use anonymous citations} and do not include
acknowledgements when submitting your papers. Papers that do not
conform to these requirements may be rejected without review.

\textbf{References}: Gather the full set of references together under
the heading {\bf References}; place the section before any Appendices,
unless they contain references. Arrange the references alphabetically
by first author, rather than by order of occurrence in the text.
Provide as complete a citation as possible, using a consistent format,
such as the one for {\em Computational Linguistics\/} or the one in the 
{\em Publication Manual of the American 
Psychological Association\/}~\cite{APA:83}.  Use of full names for
authors rather than initials is preferred.  A list of abbreviations
for common computer science journals can be found in the ACM 
{\em Computing Reviews\/}~\cite{ACM:83}.

The \LaTeX{} and Bib\TeX{} style files provided roughly fit the
American Psychological Association format, allowing regular citations, 
short citations and multiple citations as described above.

{\bf Appendices}: Appendices, if any, directly follow the text and the
references (but see above).  Letter them in sequence and provide an
informative title: {\bf Appendix A. Title of Appendix}.

\subsection{Footnotes}

{\bf Footnotes}: Put footnotes at the bottom of the page and use 9
points text. They may be numbered or referred to by asterisks or other
symbols.\footnote{This is how a footnote should appear.} Footnotes
should be separated from the text by a line.\footnote{Note the line
separating the footnotes from the text.}

\subsection{Graphics}

{\bf Illustrations}: Place figures, tables, and photographs in the
paper near where they are first discussed, rather than at the end, if
possible.  Wide illustrations may run across both columns.  Color
illustrations are discouraged, unless you have verified that  
they will be understandable when printed in black ink.

{\bf Captions}: Provide a caption for every illustration; number each one
sequentially in the form:  ``Figure 1. Caption of the Figure.'' ``Table 1.
Caption of the Table.''  Type the captions of the figures and 
tables below the body, using 11 point text.


\section{XML conversion and supported \LaTeX\ packages}

Following ACL 2014 we will also we will attempt to automatically convert 
your \LaTeX\ source files to publish papers in machine-readable 
XML with semantic markup in the ACL Anthology, in addition to the 
traditional PDF format.  This will allow us to create, over the next 
few years, a growing corpus of scientific text for our own future research, 
and picks up on recent initiatives on converting ACL papers from earlier 
years to XML. 

We encourage you to submit a ZIP file of your \LaTeX\ sources along
with the camera-ready version of your paper. We will then convert them
to XML automatically, using the LaTeXML tool
(\url{http://dlmf.nist.gov/LaTeXML}). LaTeXML has \emph{bindings} for
a number of \LaTeX\ packages, including the ACL 2015 stylefile. These
bindings allow LaTeXML to render the commands from these packages
correctly in XML. For best results, we encourage you to use the
packages that are officially supported by LaTeXML, listed at
\url{http://dlmf.nist.gov/LaTeXML/manual/included.bindings}





\section{Translation of non-English Terms}

It is also advised to supplement non-English characters and terms
with appropriate transliterations and/or translations
since not all readers understand all such characters and terms.
Inline transliteration or translation can be represented in
the order of: original-form transliteration ``translation''.

\section{Length of Submission}
\label{sec:length}

Long papers may consist of up to 8 pages of content, plus two extra
pages for references. Short papers may consist of up to 4 pages of
content, plus two extra pages for references.  Papers that do not
conform to the specified length and formatting requirements may be
rejected without review.



\section*{Acknowledgments}

The acknowledgments should go immediately before the references.  Do
not number the acknowledgments section. Do not include this section
when submitting your paper for review.

% include your own bib file like this:
%\bibliographystyle{acl}
%\bibliography{acl2015}



\begin{thebibliography}{}

\section*{Bibliograpy}
\bibitem[\protect\citename{Jindal and Liu}2008]{Jindal:08}
Nihar Jindal and Bing Liu.
\newblock 2008.
\newblock Opinion spam and analysis. 
\newblock{\em Proceeding of the International Conference on Web Search and Web Data Mining}, 219-230. ACM.

\bibitem[\protect\citename{Fornaciari and Poesio}2014]{Forn:14}
Tommaso Fornaciari and Massimo Poesio.
\newblock 2014.
\newblock Using Web-Intelligence for Excavating the
	Emerging Meaning of Target-Concepts.
\newblock{\em Proceedings of the 14th Conference of the European Chapter of the Association for Computational Linguistics}, 279-287. Gothenburg, Sweden. April. Association for Computational Linguistics.
\bibitem[\protect\citename{Ott et al.}2011]{Ott:11}
Myle Ott, Yejin Choi, Claire Cardie, and Jeffrey T. Hancock.
\newblock 2011.
\newblock  Finding deceptive opinion spam by any stretch of the imagination
\newblock{\em Proceedings of the 49th Annual Meeting of the Association for Computational Linguistics: Human Language Technologies}, 309-319. Portland, Oregon. June. Association for Computational Linguistics.
\bibitem[\protect\citename{Feng et al.}2012a]{Feng:12a}
Song Feng, Ritwik Banerjee, and Yejin Choi
\newblock 2012a.
\newblock Syntactic stylometry for deception detection.
\newblock {\em Pr Proceedings of the 50th Annual Meeting of the Association for Computational Linguistics (Volume 2: Short Papers)}, 171-175. Jeju Island, Korea. Association for Computational Lingusitics.

\end{thebibliography}



%\begin{thebibliography}{}
%
%\bibitem[\protect\citename{Aho and Ullman}1972]{Aho:72}
%Alfred~V. Aho and Jeffrey~D. Ullman.
%\newblock 1972.
%\newblock {\em The Theory of Parsing, Translation and Compiling}, volume~1.
%\newblock Prentice-{Hall}, Englewood Cliffs, NJ.
%
%\bibitem[\protect\citename{{American Psychological Association}}1983]{APA:83}
%{American Psychological Association}.
%\newblock 1983.
%\newblock {\em Publications Manual}.
%\newblock American Psychological Association, Washington, DC.
%
%\bibitem[\protect\citename{{Association for Computing Machinery}}1983]{ACM:83}
%{Association for Computing Machinery}.
%\newblock 1983.
%\newblock {\em Computing Reviews}, 24(11):503--512.
%
%\bibitem[\protect\citename{Chandra \bgroup et al.\egroup }1981]{Chandra:81}
%Ashok~K. Chandra, Dexter~C. Kozen, and Larry~J. Stockmeyer.
%\newblock 1981.
%\newblock Alternation.
%\newblock {\em Journal of the Association for Computing Machinery},
%  28(1):114--133.
%
%\bibitem[\protect\citename{Gusfield}1997]{Gusfield:97}
%Dan Gusfield.
%\newblock 1997.
%\newblock {\em Algorithms on Strings, Trees and Sequences}.
%\newblock Cambridge University Press, Cambridge, UK.
%
%\end{thebibliography}

\end{document}
